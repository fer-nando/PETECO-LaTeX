\documentclass[a4paper]{article}

%% Pacotes %%
\usepackage[brazil]{babel}
\usepackage[latin1]{inputenc}
\usepackage[T1]{fontenc}
\usepackage{verbatim}
\usepackage{hyperref}


%% T�tulo %%
\title{Minicurso de \LaTeX}
\author{PET-ECO}
\date{2012/1}

%% Comandos personalizados %%
\newcommand{\mitem}[2]{
	\begin{minipage}[c]{.5\textwidth}
	\item #1
	\end{minipage}
	\fbox{\begin{minipage}[c]{.5\textwidth}
	\small \tt #2
	\end{minipage}}
}

%% In�cio do documento %%
\begin{document}

\maketitle
\thispagestyle{empty}

\section*{Roteiro:}

\

\begin{enumerate}
	\mitem
		{O que � o \LaTeX}
		{\LaTeX\ � um processador de textos.}
	\mitem
		{Como usar: Windows e Linux
		\begin{enumerate}
			\item O que percisa instalar
			\item Editores
		\end{enumerate}
		}
		{	Windows: \emph{MikTeX}\\
			Linux: \emph{texlive}
				
			\hrulefill
			
			Windows: TexWorks, TeXnicCenter, TexMaker, \ldots \\
			Linux: Kyle, TexMaker, Vim, Emacs  \ldots
		}			
	\mitem
		{Hello World!}
		{\verbatiminput{helloworld.tex}}%a
	\mitem
		{Acentua��o e caracteres especiais}
		{\# \ \$ \ \% \ \^{} \quad \& \ \_ \ \{ \ \} \ \~{} \quad \textbackslash
		\verbatiminput{caracteres_especiais.txt}
		\'{a} \quad \`{a} \quad \^{a} \quad \~{a} \quad \"{a} \quad \c{c}}
	\mitem
		{Pacotes}
		{Pacotes adicionam novas funcionalidades ao \LaTeX \\
		Ex: \emph{*inputenc}, \emph{*fontenc}, \emph{*babel}, \emph{hyperref}, \emph{geometry}, \emph{setspace},\ \ldots }
	\mitem
		{Espa�amento e alinhamento do texto}
		{\verbatiminput{alinhamento.txt}
		center, flushright, flushleft}
	\mitem
		{Fontes: tipo, tamanho, forma, \ldots}
		{\verbatiminput{fontes.txt}}
	\mitem
		{Cap�tulos, se��es, \ldots}
		{\verbatiminput{capitulos.txt}
		\textbackslash{}tableofcontents}
	\mitem
		{Listas}
		{itemize, enumerate, description}
	\mitem
		{Modo matem�tico}
		{pacotes \emph{amsmath} e \emph{amssymb}
		\verbatiminput{modomatematico.txt}}
	\mitem
		{Equa��es matem�ticas}
		{equation, equation*, eqnarray, eqnarray*\\
		\textbackslash{}nonumber}
	\mitem
		{Tabelas}
		{\textbackslash{}begin\{tabular\}\{\emph{formata��o}\} \ldots\\
		Formata��o:\\
		c $\rightarrow$ centro\\
		l $\rightarrow$ esquerdo\\
		r $\rightarrow$ direito\\
		p\{\emph{larg}\} $\rightarrow$ largura fixa,\\
		@\{\emph{sep}\} $\rightarrow$ usa \emph{sep} entre colunas\\
		\textbar $\rightarrow$ linha vertical entre colunas\\
		Elementos:\\
		\& $\rightarrow$ separador de coluna\\
		\textbackslash{}hline ou \textbackslash{}cline\{x-y\} $\rightarrow$ linha hor.\\
		\textbackslash{}multicolumn\{\emph{n}\}\{\emph{cols}\}\{\emph{text}\}
		}
	\mitem
		{Figuras}
		{pacote \emph{graphicx}\\
		\textbackslash{}includegraphics[\emph{tam}]\{\emph{arquivo}\}}
	\mitem
		{Ambientes flutuantes}
		{\textbackslash{}begin\{figure\}[\emph{pos}]\\
		\textbackslash{}begin\{table\}[\emph{pos}]\\
		\textbackslash{}caption\{\}
		Posi��o:\\
		h $\rightarrow$ aqui~\cite{maze} \\
		t $\rightarrow$ topo~\cite{regraMao} \\
		b $\rightarrow$ fim\\
		p $\rightarrow$ p�gina especial\\
		! $\rightarrow$ for�ado
		}
	\mitem
		{Minipage}
		{Inserir figuras, texto ou tabelas alinhados lado a lado.
		\textbackslash{}minipage[\emph{pos}]\{\emph{tam}\}\{\emph{texto}\}}
	\mitem
		{Verbatim}
		{package \emph{verbatim}\\
		\textbackslash{}verb\textbar{}texto{}\textbar\\
		\textbackslash{}begin\{verbatim\} ... \textbackslash{}end\{verbatim\}\\
		\textbackslash{}verbatiminput\{arquivo\} }
	\mitem
		{Cita��es, notas de rodap�, \ldots}
		{\textbackslash{}cite\{\} $\rightarrow$ bib\\
		\textbackslash{}ref\{\} $\rightarrow$ \textbackslash{}label\{\} \\
		\textbackslash{}pageref\{\} $\rightarrow$ p�gina\\
		\textbackslash{}footnote\{\} \\[6pt]
		\textbackslash{}listoffigures\\
		\textbackslash{}listoftables }
	\mitem
		{Refer�ncias bibliogr�ficas\\
		(BibTeX+JabRef)}
		{\textbackslash{}bibliography\{file\}\\
		\textbackslash{}bibliographystyle\{plain\}}
\end{enumerate}

%% Fim do documento %%
\end{document}
